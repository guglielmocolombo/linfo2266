\subsection{Strawberry Problem}

In this approach, a column represents a single greenhouse placed at a precise location on the field.
Formally, each greenhouse is a pattern $p$ defined by a matrix $G_{p}$ where
\begin{equation}
    G_{pij} = \left\{ \begin{array}{lcl}
        1 && \text{if greenhouse }p\text{ covers cell }(i,j), \\
        0 && \text{otherwise}.
    \end{array} \right.
\end{equation}
for all $1 \le i \le A$ and $1 \le j \le B$.
We suppose that $G_{pij}$ covers a rectangular area of the field.

We associate a variable $x_p$ to each pattern to decide whether it is taken in the solution or not.
This allows us to write the objective function and the constraints of the problem:
\begin{align}
    \min &\sum_p x_p \left( C_g + \sum_{i=1}^A \sum_{j=1}^B G_{pij} C_u \right) & \label{objective}\\
    &\sum_p x_p G_{pij} \le 1 & 1 \le i \le A, 1 \le j \le B \label{no-overlap}\\
    &\sum_p x_p G_{pij} \ge s(i,j) & 1 \le i \le A, 1 \le j \le B \label{cover-strawberries}
\end{align}

In \cref{objective}, for each selected greenhouse, we add the fixed cost $C_g$ plus the unit cost $C_u$ for each cell covered by it.
\Cref{no-overlap} states that no two selected greenhouses can both cover a cell.
Similarly, \cref{cover-strawberries} ensures that all strawberries are covered by a greenhouse.

Since the number of possible greenhouses is very large, we will start with a small number of them
and then introduce new promising patterns, based on intermediate solutions.
To initialize the algorithm, a simple idea would be to have one greenhouse covering each cell with a strawberry.

We now formulate the \textit{pricing problem} which will find new interesting columns, if any.
A column will be able to improve a current solution if it has a \textit{negative reduced cost}.
As in the Simplex algorithm, the goal of the pricing problem is to find the column with the most negative reduced cost.
\bigbreak
\noindent\fbox{\begin{minipage}{\textwidth}
Given a linear program in matrix form:
\begin{align}
    \min~ &c^\top x\\
    &Ax \le b\\
    &x \ge 0
\end{align}
and its dual linear program:
\begin{align}
    \max~ &b^\top y\\
    &A^\top x \ge c\\
    &y \ge 0
\end{align}
the reduced cost vector is given by:
\begin{equation}
    \overline{c} = c - A^\top y
\end{equation}
\end{minipage}}
\bigbreak
For a column $p$ of our problem, we obtain the following reduced cost:
\begin{align}
    \overline{c_p} &= C_g + \sum_{i=1}^A \sum_{j=1}^B G_{pij} C_u - \sum_{i=1}^A \sum_{j=1}^B \alpha_{ij} G_{pij} - \sum_{i=1}^A \sum_{j=1}^B \beta_{ij} (- G_{pij}) \\
    &= C_g + \sum_{i=1}^A \sum_{j=1}^B G_{pij} \left( C_u - \alpha_{ij} + \beta_{ij} \right)
\end{align}

Finally, the pricing problem is given by:
\begin{equation}
    \min C_g + \sum_{i=1}^A \sum_{j=1}^B G_{ij} \left( C_u - \alpha_{ij} + \beta_{ij} \right)
\end{equation}
with $G_{ij}$ the decision variables and $G$ describing a valid greenhouse.
This problem resumes to the Maximum Sum Rectangle problem (try to find how)
which can be solved with a well-known dynamic programming algorithm based on Kadane's algorithm.
